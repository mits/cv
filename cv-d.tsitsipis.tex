%%%%%%%%%%%%%%%%%%%%%%%%%%%%%%%%%%%%%%%%%%%%%%%%%%%%%%%%%%%%%%%%%%%%%%%%
%%%%%%%%%%%%%%%%%%%%%% Simple LaTeX CV Template %%%%%%%%%%%%%%%%%%%%%%%%
%%%%%%%%%%%%%%%%%%%%%%%%%%%%%%%%%%%%%%%%%%%%%%%%%%%%%%%%%%%%%%%%%%%%%%%%

%%%%%%%%%%%%%%%%%%%%%%%%%%%%%%%%%%%%%%%%%%%%%%%%%%%%%%%%%%%%%%%%%%%%%%%%
%% NOTE: If you find that it says                                     %%
%%                                                                    %%
%%                           1 of ??                                  %%
%%                                                                    %%
%% at the bottom of your first page, this means that the AUX file     %%
%% was not available when you ran LaTeX on this source. Simply RERUN  %%
%% LaTeX to get the ``??'' replaced with the number of the last page  %%
%% of the document. The AUX file will be generated on the first run   %%
%% of LaTeX and used on the second run to fill in all of the          %%
%% references.                                                        %%
%%%%%%%%%%%%%%%%%%%%%%%%%%%%%%%%%%%%%%%%%%%%%%%%%%%%%%%%%%%%%%%%%%%%%%%%

%%%%%%%%%%%%%%%%%%%%%%%%%%%% Document Setup %%%%%%%%%%%%%%%%%%%%%%%%%%%%

% Don't like 10pt? Try 11pt or 12pt
\documentclass[11pt]{article}

% This is a helpful package that puts math inside length specifications
\usepackage{calc}

% Simpler bibsection for CV sections
% (thanks to natbib for inspiration)
\makeatletter
\newlength{\bibhang}
\setlength{\bibhang}{1em}
\newlength{\bibsep}
 {\@listi \global\bibsep\itemsep \global\advance\bibsep by\parsep}
\newenvironment{bibsection}
    {\minipage[t]{\linewidth}\list{}{%
        \setlength{\leftmargin}{\bibhang}%
        \setlength{\itemindent}{-\leftmargin}%
        \setlength{\itemsep}{\bibsep}%
        \setlength{\parsep}{\z@}%
        }}
    {\endlist\endminipage}
\makeatother

% Layout: Puts the section titles on left side of page
\reversemarginpar

%
%         PAPER SIZE, PAGE NUMBER, AND DOCUMENT LAYOUT NOTES:
%
% The next \usepackage line changes the layout for CV style section
% headings as marginal notes. It also sets up the paper size as either
% letter or A4. By default, letter was used. If A4 paper is desired,
% comment out the letterpaper lines and uncomment the a4paper lines.
%
% As you can see, the margin widths and section title widths can be
% easily adjusted.
%
% ALSO: Notice that the includefoot option can be commented OUT in order
% to put the PAGE NUMBER *IN* the bottom margin. This will make the
% effective text area larger.
%
% IF YOU WISH TO REMOVE THE ``of LASTPAGE'' next to each page number,
% see the note about the +LP and -LP lines below. Comment out the +LP
% and uncomment the -LP.
%
% IF YOU WISH TO REMOVE PAGE NUMBERS, be sure that the includefoot line
% is uncommented and ALSO uncomment the \pagestyle{empty} a few lines
% below.
%

%% Use these lines for letter-sized paper
%\usepackage[paper=letterpaper,
            %includefoot, % Uncomment to put page number above margin
%            marginparwidth=1.2in,     % Length of section titles
%            marginparsep=.05in,       % Space between titles and text
%            margin=1in,               % 1 inch margins
%            includemp]{geometry}

%% Use these lines for A4-sized paper
\usepackage[paper=a4paper,
%            %includefoot, % Uncomment to put page number above margin
            marginparwidth=30.5mm,    % Length of section titles
            marginparsep=1.5mm,       % Space between titles and text
            margin=25mm,              % 25mm margins
            includemp]{geometry}

%% More layout: Get rid of indenting throughout entire document
\setlength{\parindent}{0in}

%% This gives us fun enumeration environments. compactitem will be nice.
\usepackage{paralist}

%for utf8, needs xelatex
%\usepackage{fontspec}
%\usepackage{xunicode}
%\usepackage{xltxtra}
%font setup
%Computer Modern Unicode
%\setromanfont{CMU Serif}
%\setsansfont{CMU Sans Serif}
%\setmonofont{CMU Typewriter Text}
%\setmainfont{CMU Serif}

%free fonts
%\setromanfont{FreeSerif}
%\setsansfont{CMU Sans}
%\setmonofont{CMU Mono}
%\setmainfont{CMU Serif}

\usepackage[english, greek]{babel}
\usepackage[utf8x]{inputenc}
\usepackage{kerkis}

%% Reference the last page in the page number
%
% NOTE: comment the +LP line and uncomment the -LP line to have page
%       numbers without the ``of ##'' last page reference)
%
% NOTE: uncomment the \pagestyle{empty} line to get rid of all page
%       numbers (make sure includefoot is commented out above)
%
\usepackage{fancyhdr,lastpage}
\pagestyle{fancy}
%\pagestyle{empty}      % Uncomment this to get rid of page numbers
\fancyhf{}\renewcommand{\headrulewidth}{0pt}
\fancyfootoffset{\marginparsep+\marginparwidth}
\newlength{\footpageshift}
\setlength{\footpageshift}
          {0.5\textwidth+0.5\marginparsep+0.5\marginparwidth-2in}
\lfoot{\hspace{\footpageshift}%
       \parbox{4in}{\, \hfill %
                    \arabic{page} \greektext / \protect\pageref*{LastPage} % +LP
%                    \arabic{page}                               % -LP
                    \hfill \,}}

% Finally, give us PDF bookmarks
\usepackage{color}
\usepackage[unicode]{hyperref}
\definecolor{darkblue}{rgb}{0.0,0.0,0.3}
\hypersetup{colorlinks,breaklinks,
            linkcolor=darkblue,urlcolor=darkblue,
            anchorcolor=darkblue,citecolor=darkblue}


%%%%%%%%%%%%%%%%%%%%%%%% End Document Setup %%%%%%%%%%%%%%%%%%%%%%%%%%%%


%%%%%%%%%%%%%%%%%%%%%%%%%%% Helper Commands %%%%%%%%%%%%%%%%%%%%%%%%%%%%

% The title (name) with a horizontal rule under it
%
% Usage: \makeheading{name}
%
% Place at top of document. It should be the first thing.
\newcommand{\makeheading}[1]%
        {\hspace*{-\marginparsep minus \marginparwidth}%
         \begin{minipage}[t]{\textwidth+\marginparwidth+\marginparsep}%
                {\large \bfseries #1}\\[-0.15\baselineskip]%
                 \rule{\columnwidth}{1pt}%
         \end{minipage}}

% The section headings
%
% Usage: \section{section name}
%
% Follow this section IMMEDIATELY with the first line of the section
% text. Do not put whitespace in between. That is, do this:
%
%       \section{My Information}
%       Here is my information.
%
% and NOT this:
%
%       \section{My Information}
%
%       Here is my information.
%
% Otherwise the top of the section header will not line up with the top
% of the section. Of course, using a single comment character (%) on
% empty lines allows for the function of the first example with the
% readability of the second example.
\renewcommand{\section}[2]%
        {\pagebreak[2]\vspace{1.3\baselineskip}%
         \phantomsection\addcontentsline{toc}{section}{#1}%
         \hspace{0in}%
         \marginpar{\raggedright \scshape #1}#2}
%\quad\pagebreak[2]}

% An itemize-style list with lots of space between items
\newenvironment{outerlist}[1][\enskip\textbullet]%
        {\begin{itemize}[#1]}{\end{itemize}%
         \vspace{-.6\baselineskip}}

% An environment IDENTICAL to outerlist that has better pre-list spacing
% when used as the first thing in a \section
\newenvironment{lonelist}[1][\enskip\textbullet]%
        {\vspace{-\baselineskip}\begin{list}{#1}{%
        \setlength{\partopsep}{0pt}%
        \setlength{\topsep}{0pt}}}
        {\end{list}\vspace{-.6\baselineskip}}

% An itemize-style list with little space between items
\newenvironment{innerlist}[1][\enskip\textbullet]%
        {\begin{compactitem}[#1]}{\end{compactitem}}

% An environment IDENTICAL to innerlist that has better pre-list spacing
% when used as the first thing in a \section
\newenvironment{loneinnerlist}[1][\enskip\textbullet]%
        {\vspace{-\baselineskip}\begin{compactitem}[#1]}
        {\end{compactitem}\vspace{-.6\baselineskip}}

% To add some paragraph space between lines.
% This also tells LaTeX to preferably break a page on one of these gaps
% if there is a needed pagebreak nearby.
\newcommand{\blankline}{\quad\pagebreak[2]}

% Uses hyperref to link DOI
\newcommand\doilink[1]{\href{http://dx.doi.org/#1}{#1}}
\newcommand\doi[1]{doi:\doilink{#1}}

%for multiline comments
\newcommand{\comments}[1]{}

%shorthands, used with pdflatex
\newcommand{\grt}{\greektext}
\newcommand{\lat}{\latintext}
%null when xelatex is used
%\newcommand{\grt}{}
%\newcommand{\lat}{}

%%%%%%%%%%%%%%%%%%%%%%%% End Helper Commands %%%%%%%%%%%%%%%%%%%%%%%%%%%

%%%%%%%%%%%%%%%%%%%%%%%%% Begin CV Document %%%%%%%%%%%%%%%%%%%%%%%%%%%%
\begin{document}

\pretolerance=3000
\hyphenpenalty=5000

\makeheading{\huge Δημήτριος Π.~Τσιτσιπής}

\section{Πληροφορίες επικοινωνίας}
%
% NOTE: Mind where the & separators and \\ breaks are in the following
%       table.
%
% ALSO: \rcollength is the width of the right column of the table
%       (adjust it to your liking; default is 1.85in).
%
%\newlength{\rcollength}\setlength{\rcollength}{6cm}%
%
%\begin{tabular}[t]{@{}p{\textwidth-\rcollength}p{\rcollength}}
\begin{tabular}{ll}
\textit{Διεύθυνση κατοικίας:} & Μαιζώνος 231 \\
                              & Πάτρα, 26222\\
\textit{Τηλέφωνο κατοικίας:}     & 2610361921\\
\textit{Τηλέφωνο γραφείου:}	& 2610962485\\
\textit{Κινητό τηλέφωνο:}     &6974605599 \\
\lat \textit{ E-mail:}  & \lat \href{mailto:mitsarionas@ece.upatras.gr}{mitsarionas@ece.upatras.gr}
\end{tabular}

\blankline

\section{Ερευνητικά Ενδιαφέροντα}
%
Δίκτυα, ενσωματωμένα συστήματα, λειτουργικά συστήματα, γραφικά, γλώσσες προγραμματισμού

\blankline

\section{Εκπαίδευση}
%
\href{http://www.upatras.gr}{\textbf{Πανεπιστήμιο Πατρών}}

\begin{outerlist}

\item[] Υποψήφιος για διδακτορικό δίπλωμα,\\
        \href{http://www.ece.upatras.gr/}
             {\textbf{Τμήμα Ηλεκτρολόγων και Τεχνολογίας Υπολογιστών}}
%             (expected graduation date: August 2010)
        \begin{innerlist}
        \item Θέμα διατριβής: \emph{Τεχνικές ασφαλείας σε ασύρματα δίκτυα αισθητήρων}
%        \item Thesis Proposal: \emph{...}
%        \item Candidacy Exam: \emph{...}
        \item Επιβλέπων:
%              \href{http://www.apel.ee.upatras.gr/the_lab/faculty/koubias_gr.htm}
                   {Σταύρος Κουμπιάς, Καθηγητής}
%        \item Area of Study: ...
        \end{innerlist}

\item[] Μ.Δ.Ε. στα
        \href{http://www.upatras.gr/hw-sw/}
             {Ολοκληρωμένα Συστήματα Υλικού και Λογισμικού},\\
        \href{http://www.ceid.upatras.gr/}
             {\textbf{Τμήμα Μηχανικών Υπολογιστών και Πληροφορικής}},\\
	    Δεκέμβριος 2009
        \begin{innerlist}
        \item Βαθμός: Άριστα (8.5)
        \item Θέμα Διπλωματικής: \emph{Υλοποίηση του πρωτοκόλλου \lat S\textsuperscript{2}RP (Secure and Scalable Rekeying Protocol)}
        \item Επιβλέπων:
              {Κωνσταντίνος Γκούτης, Καθηγητής}
        \end{innerlist}

\item[] Πτυχίο,\\
        \href{http://www.ece.upatras.gr/}
             {\textbf{Τμήμα Ηλεκτρολόγων Μηχανικών και Τεχνολογίας Υπολογιστών}},\\ Νοέμβριος 2007
        \begin{innerlist}
        \item Βαθμός: Λίαν Καλώς (7.04)
        \item Θέμα Διπλωματικής: \emph{Σχεδιασμός κωδικοποιητή ακολουθιών κινούμενων εικόνων \lat(video coder)
\grt βασισμένου στο μετασχηματισμό \lat wavelet}
        \item Επιβλέπων:
              {Στουραΐτης Θάνος, Καθηγητής}
        \end{innerlist}

\end{outerlist}

\blankline

\section{Διακρίσεις}
%
\href{http://www.epy.gr/}{Ελληνική Εταιρεία Επιστημόνων και Επαγγελματιών Πληροφορικής και Επικοινωνιών (ΕΠΥ)}
\begin{innerlist}
\item Βράβευση σε πανελλήνιους διαγωνισμούς Πληροφορικής τα έτη 1995-96, 1996-97, 1997-98
\item Βράβευση για την 4η θέση σε πανελλήνιο διαγωνισμό Πληροφορικής διοργανωμένο από την ΕΠΥ το 2000-2001 και συμμετοχή στην Εθνική Ομάδα Πληροφορικής.
\end{innerlist}
%Βραβείο από το υπουργείο παιδείας για άριστο μέσο όρο (πάνω από 18.5) για καθένα από τα χρόνια μαθητείας (1995-96, 1996-97, 1997-98, 1998-99, 1999-00,   2000-01)

\blankline

\section{Δημοσιεύσεις}
%Σε συνέδρια\\
\lat
\begin{bibsection}
%\begin{outerlist}
    \item Anastasopoulos, A.; Tsitsipis, D,; Giannoulis, S.; Koubias, S.; 
Implementation and Evaluation of a Hybrid Network utilizing TinyOS-based systems and Ethernet
 \emph{(2007) IEEE Symposium on Emerging Technologies and Factory Automation, ETFA,} art. no. 4416801, pp. 441-447.
%\end{outerlist}
\end{bibsection}
\grt

\blankline

\section{Διδακτική Εμπειρία}
\textbf{Τμήμα Ηλεκτρολόγων Μηχανικών και Τεχνολογίας Υπολογιστών}, \\Πανεπιστήμιο~Πατρών
\begin{outerlist}
\item[] \hfill Μάρτιος~2010 - Σήμερα \\ Επικουρική Διδασκαλία στο εργαστήριο του μαθήματος ``Ψηφιακά ολοκληρωμένα κυκλώματα και συστήματα''
\item[] \hfill Οκτώβριος~2009 - Φεβρουάριος~2010 \\ Επικουρική Διδασκαλία στο εργαστήριο του μαθήματος ``Ψηφιακά ολοκληρωμένα κυκλώματα και συστήματα''
\end{outerlist}

\blankline

\textbf{Τμήμα Μηχανικών Ηλεκτρονικών Υπολογιστών και Πληροφορικής}, \\Πανεπιστήμιο~Πατρών
\begin{outerlist}
\item[] \hfill~Μάρτιος~2008~-~Ιούνιος~2008\\ Δημιουργία εκπαιδευτικού και διδακτικού υλικού του εργαστηριακού μαθήματος ``Βασικά Ηλεκτρονικά``
\item[] \hfill Μάρτιος~2008 - Ιούνιος~2008\\ Επικουρική διδασκαλία στο εργαστηρίο του μαθήματος ''Τεχνολογία και Αρχιτεκτονική Υπολογιστών''
\item[] \hfill Νοέμβριος~2007 - Φεβρουάριος~2008\\ Επικουρική διδασκαλία στο εργαστηρίο του μαθήματος ``Μικροηλεκτρονική``
\end{outerlist}

\blankline

%\section{Επαγγελματική Εμπειρία}
%
%Ανάπτυξη προγράμματος διαχείρισης δρομολογίων βυτιοφόρων για Δ.Ε.Υ.Α.

%\blankline

\section{Συμμετοχή σε διαγωνισμούς}
Ολυμπιάδα Προγραμματισμού \lat( International Olympiad in Informatics (IOI)), \grt Ελσίνκι, 2001.\\
Βαλκανιάδα Προγραμματισμού \lat(Balkan Olympiad in Informatics (BOI)), \grt Δυρράχιο, 2001.\\
\lat South European Informatics Championship \grt στο Βελιγράδι το 2001.

\blankline

\section{Τεχνικά Προσόντα}
%
 \grt Προγραμματισμός: \lat C, C++,C\#, Assembly, Python, Java, Pascal, Perl, UNIX
        shell scripting, GNU make, SQL, Revision Control Systems (Git, Mercurial, CVS, SVN) \grt και άλλα

\blankline

\grt Σχεδιασμός Υλικού: \lat VHDL, SystemC

\blankline

\grt Επιστημονικός Υπολογισμός: \lat Matlab

\blankline

\grt Έλεγχος και Εξομοίωση Συστημάτων: \lat Simulink, Stateflow

\blankline

\grt Ενσωματωμένα Συστήματα: \lat Contiki OS, TinyOS, TelosB platorm, Crossbow IRIS platorm, Texas Instruments C67xx DSP, Analog Devices Blackfin DSP

\blankline

%\item [] Information Technology: Networking (UDP, TCP, ARP, DNS, Dynamic
%        routing), Service (Apache, SQL, MediaWiki, POP, IMAP, SMTP,
%        application-specific daemon design)
%\blankline

\lat Computer-Aided Design: PSpice (OrCad, Cadence), Design Architect (Mentor Graphics), Leonardo Spectrum (Mentor Graphics), 
ModelSim (Mentor Graphics), MicroWind (Insa), AutoCAD

\blankline

\grt Λειτουργικά Συστήματα: \lat Linux, BSD, Solaris, QNX, Microsoft Windows, DOS

\blankline

\grt Εφαρμογές: \lat \TeX{} (\LaTeX{}, B\textsc{ib}\TeX{}), \grt εφαρμογές γραφείου \lat (Microsoft Office, OpenOffice), Vim

\blankline

\blankline

\grt
\section{Λοιπά προσόντα}
\begin{lonelist}
 \item [] Γλώσσες
\begin{innerlist}
 \item [] Αγγλικά \lat(Certificate of Proficiency in English, - Cambridge \grt και \lat Michigan) \grt
 \item [] Γερμανικά \lat (Zertifikat Deutch - Goethe Insitut)
\end{innerlist}
\end{lonelist}

\blankline

\section{Άλλα ενδιαφέροντα}
\grt Τεχνολογία, \lat PC Games, \grt ταξίδια, τέννις, σκι, ποδηλασία, μουσική

\end{document}

%%%%%%%%%%%%%%%%%%%%%%%%%% End CV Document %%%%%%%%%%%%%%%%%%%%%%%%%%%%%
